%%%%%%%%%%%%%%%%%%%%%%%%%%%%%% Preamble
\documentclass[11pt]{article}
\setlength{\parskip}{\baselineskip}%
\setlength{\parindent}{0pt}%
\usepackage{amsmath,amssymb,amsthm,physics,graphicx,titling}
\newcommand{\subtitle}[1]{%
  \posttitle{%
    \par\end{center}
    \begin{center}\large#1\end{center}
    \vskip0.5em}%
}
\newtheorem{theorem}{Theorem}[section]
\newtheorem*{remark}{Remark}
\newcommand{\boxthis}[1]{\noindent\fbox{%
    \parbox{\textwidth}{%
        #1
    }%
}
}
\begin{document}

%%%%%%%%%%%%%%%%%%%%%%%%%%%%%% Heading
	\title{Ma 2 - Problem Set 1}
	\author{Yovan Badal}
	\date{09/27/2017}
	\maketitle
	
%%%%%%%%%%%%%%%%%%%%%%%%%%%%%% Body
\section{Continuous fixed thirty-year mortgage}
\boxthis{A mortgage is a loan, usually for buying
a house, in which a borrower makes a fixed payment over the lifetime of the loan, until
the balance of the loan is reduced to zero. Usually in a real mortgage, the payment is
made monthly and the interest is compounded monthly. In this problem, we consider a
mortgage in which the payment is made as a continuous stream and so that the interest
compounds continuously. We will denote the balance of the loan at time t (in years) as
y(t). Then y(t) will be the solution given in lecture 1 to the differential equation
\[
\dv{y}{t} = ry - p,
\]
where r is a constant representing the annual rate of interest, and p is a constant representing the rate of payment per year, satisfying the initial condition
\[
y(0) = M,
\]
where M is the (nonzero) initial amount of the loan. Find a formula for p in terms of r and
M so that the balance will also satisfy y(30) = 0. At any time, the rate at which interest is
being paid on the loan is ry(t). Thus the amount of the payment being applied to principal
is $q(t) = p - ry(t)$. Without using the solution you obtained, find a differential equation
(hopefully as simple as possible) satisfied by the quantity q(t). Plugging in your solution
for y(t), determine the initial condition q(0) and verify that q(0) satisfies the differential
equation that you found.}

\end{document}