%%%%%%%%%%%%%%%%%%%%%%%%%%%%%% Preamble
\documentclass[11pt]{article}
\setlength{\parskip}{\baselineskip}%
\setlength{\parindent}{0pt}%
\usepackage{amsmath,amssymb,amsthm,physics,graphicx,titling}
\newcommand{\subtitle}[1]{%
  \posttitle{%
    \par\end{center}
    \begin{center}\large#1\end{center}
    \vskip0.5em}%
}

\begin{document}

%%%%%%%%%%%%%%%%%%%%%%%%%%%%%% Heading
	\title{Ma 2 - Problem Set 3}
	\author{Yovan Badal}
	\date{10/14/2017}
	\maketitle
	
%%%%%%%%%%%%%%%%%%%%%%%%%%%%%% Body
\section*{Problem 1}
An obvious candidate for a trial function that might satisfy our differential equation is $\lambda = -2; u(t) = e^{\lambda t}$. We try this with our equation:
\begin{align*}
\lambda^2 e^{\lambda t} + 4 \lambda e^{\lambda t} + 4e^{\lambda t} &= 0 \\
\lambda^2 e^{\lambda t} + 4 \lambda + 4 = 0 \\
(\lambda +2)^2 = 0
\end{align*}
We can see that $u(t)=e^{-2t}$ works as a solution. Now, we try a solution of the form $u(t) = te^{\lambda t}$, since both the first and second derivatives will give us a linear combination of $e^{\lambda t}$ and $te^{\lambda t}$.
\begin{align*}
\lambda^2 t e^{\lambda t} + 2 \lambda t e^{\lambda t} + 4 \lambda t e^{\lambda t} + 4e^{\lambda t} + 4te^{\lambda t} &= 0 \\
(\lambda^2 + 4\lambda +4)te^{\lambda t} + (2\lambda +4)e^{\lambda t} &= 0
\end{align*}
We can again see that $\lambda = -2; u(t) = te^{-2t}$ works.

Clearly, we now have two linearly independent solutions to our differential equations. We therefore have a general solution:
\[
u(t) = C_1 e^{-2t} + C_2 te^{-2t}
\] and 
\[
\dv{u}{t} = -2C_1e^{-2t} - 2C_2te^{-2t} + C_2 e^{-2t}
\]
in which we can plug in our initial conditions. This gives:
\begin{align*}
C_1 &= 5 \\
-2C_1 + C_2 &= 7 \\
C_2 &= 17
\end{align*}
Which gives us the solution:
\[
u(t) = 5e^{-2t} + 17te^{-2t}
\]

\section*{Problem 2}
First, we set up the matrix $A$ for the homogeneous differential equation as in Lecture 6.
\[
A = \mqty(0 & 1 \\ -2 & -3)
\]
We can now find the eigenvalues of this matrix:
\begin{align*}
det(A-\lambda I) &= 0 \\
\mdet{-\lambda & 1 \\ -2 & -3-\lambda} &= 0 \\
\lambda(\lambda + 3) + 2 &= 0 \\
(\lambda + 2)(\lambda + 1) &= 0
\end{align*}
We thus find $\lambda_1 = -1; \lambda_2 = -2$. We will proceed by finding a special solution for the equation for one forcing term at a time, then taking a linear combination of our solutions, as per Lecture 8.

We can factor our differential operator $L=D^2 + 3D + 2$ as $L = (D-\lambda_1)(D-\lambda_2) = (D+1)(D+2)$. 

Therefore, for an equation of the form $Lu=e^{\rho t}$, we can use the guess $u(t) = Ce^{\rho t}$ and the result $Lu = C(\rho - \lambda_1)(\rho - \lambda_2)e^{\rho t}$ to find a specific solution to $Lu = e^{3t}$. From above, we have $C = \frac{1}{(\rho - \lambda_1)(\rho - \lambda_2)} = \frac{1}{20}; u(t) = \frac{1}{20} e^{3t}$.

For an equation of the form $Lu = te^{\rho t}$, we can use the guess $u(t) = C_1 te^{\rho t} + C_2 e^{\rho t}$ and the result $Lu = C_1(\rho - \lambda_1)(\rho - \lambda_2)te^{\rho t} + (C_1[(\rho - \lambda_1) + (\rho - \lambda_2)] + C_2(\rho - \lambda_1)(\rho - \lambda_2))e^{\rho t}$ to find a specific solution to $Lu = te^{4t}$. From above, we have $C_1 = \frac{1}{(\rho - \lambda_1)(\rho - \lambda_2)} = \frac{1}{30}$ and $C_2 = -C_1 (\frac{1}{\rho - \lambda_1} + \frac{1}{\rho - \lambda_2}) = \frac{-11}{900}; u(t) = -\frac{11}{900} te^{4t}$.

We now need to take care of the equation of the form $Lu = t^2 e^{-2t}$. We observe that the value of the exponent is $\lambda_2 = -2$. This means that it's useless to have any term of the form $t^2 e^{\lambda_2 t}$ in our guess function, because the $(D - \lambda_2)$ operator kills it. We figure a more sophisticated guess is appropriate, and choose $u(t) = C_1 t^3 e^{\lambda_2 t} + C_2 t^2 e^{\lambda_2 t} + C_3 t e^{\lambda_2 t}$ in the hope that each term cleans up the previous one's mess leaving behind only the $t^2 e^{\lambda_2 t}$ term.

We apply $(D - \lambda_2)$ and obtain
\[
(D - \lambda_2)u = C_1 (3t^2 e^{\lambda_2 t}) + C_2(2t e^{\lambda_2 t}) + C_3 (e^{\lambda_2 t})
\]
We apply $(D - \lambda_1)$ and obtain
\begin{align*}
(D - \lambda_1)(D - \lambda_2)u &= 3C_1(\lambda_2 t^2 e^{\lambda_2 t} + 2te^{\lambda_2 t} - \lambda_1 t^2 e^{\lambda_2 t}) + \\
& 2C_2(\lambda_2 t e^{\lambda_2 t} + e^{\lambda_2 t} - \lambda_1 te^{\lambda_2 t}) + C_3(\lambda_2 e^{\lambda_2 t} - \lambda_1 e^{\lambda_2 t}) \\
&= 3C_1 (\lambda_2 - \lambda_1)t^2 e^{\lambda_2 t} + \\
& (6C_1 + 2C_2 \big[\lambda_2 - \lambda_1\big])t e^{\lambda_2 t} + (2C_2 + C_3 \big[\lambda_2 - \lambda_2 \big])e^{\lambda_2 t}
\end{align*}
In order to make the appropriate terms appear and disappear, we set:
\begin{align*}
C_1 &= \frac{1}{3(\lambda_2 - \lambda_1)}  = -\frac{1}{3} \\
C_2 &= \frac{-3C_1}{\lambda_2 - \lambda_1} = -1\\
C_3 &= \frac{-2C_2}{\lambda_2 - \lambda_2} = -2
\end{align*}
and obtain the solution
\[
u(t) = -\frac{1}{3} t^3 e^{-2t} - t^2 e^{-2t} - 2t e^{-2t}.
\]
By linearity, we then have a special solution for our original differential equation:
\[
u(t) = \frac{1}{20} e^{3t} - \frac{11}{900} te^{4t} -\frac{1}{3} t^3 e^{-2t} - t^2 e^{-2t} - 2t e^{-2t}
\]

\section*{Problem 3}
First, we set up the matrix $A$ for the homogeneous differential equation as in Lecture 6.
\[
A = \mqty(0 & 1 \\ -2 & -2)
\]
We can now find the eigenvalues of this matrix:
\begin{align*}
det(A-\lambda I) &= 0 \\
\mdet{-\lambda & 1 \\ -2 & -2-\lambda} &= 0 \\
\lambda(\lambda + 2) + 2 &= 0 \\
\lambda^2 + 2\lambda +2 &= 0
\end{align*}
We thus find $\lambda_1 = i - 1; \lambda_2 = -i - 1$. We can factor our differential operator $L=D^2 + 2D + 2$ as $L = (D-\lambda_1)(D-\lambda_2)$.

To make our algebra nicer, we will start by solving the following equation:
\[
Lu_1 = te^{(i-1)t} = te^{-t} (isin t + cos t).
\]
We observe that value of the exponent is $\lambda_1 = i-1$. This means that it's useless to have any term of the form $t^2 e^{\lambda_1 t}$ in our guess function, because the $(D - \lambda_1)$ operator kills it. We figure a more sophisticated guess is appropriate, and choose $u_1(t) = C_1 t^2 e^{\lambda_1 t} + C_2 t e^{\lambda_1 t}$ in the hope that each term cleans up the previous one's mess leaving behind only the $t e^{\lambda_1 t}$ term.

We apply $(D - \lambda_1)$ and obtain:
\[
(D-\lambda_1)u_1 = C_1 (2t e^{\lambda_1 t}) + C_2 (e^{\lambda_1 t})
\]
We apply $(D - \lambda_2)$ and obtain
\begin{align*}
(D - \lambda_2)(D - \lambda_1)u &= 2C_1(\lambda_1 te^{\lambda_1 t} + e^{\lambda_1 t} - \lambda_2 te^{\lambda_1 t}) + C_2(\lambda_1 e^{\lambda_1 t} - \lambda_2 e^{\lambda_1 t} \\
&= 2C_1(\lambda_1 - \lambda_2)te^{\lambda_1 t} + \big[2C_1 + C_2 (\lambda_1 - \lambda_2) \big] e^{\lambda_1 t}
\end{align*}
In order to make the appropriate terms appear and disappear, we set:
\begin{align*}
C_1 &= \frac{1}{2(\lambda_1 - \lambda_2)}  = -\frac{i}{4} \\
C_2 &= \frac{-2C_1}{\lambda_1 - \lambda_2} = \frac{1}{4}
\end{align*}
and obtain the solution:
\begin{align*}
u_1(t) &= -\frac{i}{4} t^2 e^{-t} (i sin t + cos t) + \frac{1}{4} te^{-t} (i sin t + cos t) \\
u_1(t) &= (\frac{1}{4} t^2 e^{-t} sin t + \frac{1}{4} te^{-t} cos t) + i(-\frac{1}{4} t^2 e^{-t} cos t + \frac{1}{4} te^{-t} sin t)
\end{align*}
We observe that the equation we need to solve is simply the real part of the equation we have solved. We know that by linearity, a solution to a differential equation with a linear combination of forcing terms is simply the corresponding linear combination of the solutions of the equation for each individual forcing term.

Also by linearity of the $L$ operator, the imaginary terms in our solution must satisfy the equation with the imaginary forcing term, and the real terms in our solution must satisfy the equation with the real forcing term. Therefore, we can obtain a special solution for the equation we need to solve by taking the real part of the solution for the equation we solved above. A special solution to the original equation is therefore:
\[
u(t) = \frac{1}{4} t^2 e^{-t} sin t + \frac{1}{4} te^{-t} cos t
\]

\section*{Problem 4}
We use the solution from Proposition 4 in the Lecture notes, since our equation $\dv[3]{u}{t} = -t^2 \dv[2]{u}{t} - \dv{u}{t} - 2017u$ is of the form required:
\[
W(t) = e^{\int_{t_0}^t p_1 (s) \dd{s}} W(t_0)
\]
Now, we first find:
\begin{align*}
W(0) &= \mdet{u_1(0) & u_2(0) & u_3(0) \\ \dv{u_1}{t}(0) & \dv{u_2}{t}(0) & \dv{u_3}{t}(0) \\ \dv[2]{u_1}{t}(0) & \dv[2]{u_2}{t}(0) & \dv[2]{u_3}{t}(0)} \\
&= \mdet{1 & 0 & 0 \\ 2016 & 1 & 0 \\ log644 & cos747 & 1} \\
&= 1\mdet{1 & 0 \\ cos747 & 1}\\
&= 1
\end{align*}
Therefore, we have:
\begin{align*}
W(t) &= e^{\int_{0}^t -t^2 (s) \dd{s}} W(0) \\
W(t) &= e^{-\frac{1}{3} t^3}
\end{align*}

\section*{Problem 5}
We know that every $n \times n$ complex matrix has a Jordan Normal Form. We will denote the Jordan Normal Form of a matrix $At$ by J. From Lecture 5, there is an invertible matrix $B$ such that:
\[
J = B At B^{-1}.
\]
Now, we recall  that $det(ABC) = det(A) det(B) det(C)$ and use the fact that $det(B^{-1}) = det(B)^{-1}$. 
From Jordan Form Exponentiation in Lecture 5,
\begin{align*}
det(e^{At}) &= det(B e^J B^{-1}) \\
&= det(B) det(e^J) det(B^{-1}) \\
&= det(e^J)
\end{align*}
Now remember that Jordan matrices are upper triangular, and consider the method we used to exponentiate Jordan matrices: each block of the Jordan matrix is exponentiated. For each block exponentiation, we obtain a sum of matrix terms, one of which is the exponentiation of the diagonal matrix consisting of the block's diagonal elements (call this matrix $M$), and the others being products of the diagonal matrix $M$ with nilpotent upper-triangular matrices (in accordance with the power series for exponentiation) which therefore have diagonal elements 0.

Clearly the diagonal elements of the exponentiation of each block will simply be the same as the diagonal elements for the exponentiation of $M$. This makes it obvious that exponentiation of a Jordan matrix generates the same diagonal elements as the exponentiation of a diagonal matrix consisting of the Jordan matrix's diagonal elements. That is, we can find the diagonal elements of the exponentiation of a Jordan matrix simply by exponentiating the diagonal elements of the matrix (as we would for a diagonal matrix).

We will now use this along with the fact that the determinant of an upper triangular matrix is the product of its diagonal elements, denoting the diagonal elements of our Jordan matrix $J$ by $jj$:
\begin{align*}
\det(e^J) &=\prod_{i=1}^n e^{j_{ii}}\\
&=e^{\sum_{i=1}^n{j_{ii}}}\\
&=e^{tr(J)}
\end{align*}
where $tr(J)$ denotes the trace of $J$. Now, we recall from Ma1b that the trace of a matrix is conserved under Gauss-Jordan operations. Therefore,
\[
tr(J) = tr(At) \implies det(e^{At}) = e^{tr(At)}.
\]
Now, clearly, if $A$ has trace zero then $tr(At) = 0 \implies det(e^{At}) = 1$ for all t.
\end{document}